% !TEX encoding = UTF-8 Unicode
% !TEX spellcheck = en_US

\documentclass[
	graybox,
	vecphys] % vectors bold face italic (vec command)
	{svmult}

\usepackage{type1cm}        % activate if the above 3 fonts are
                            % not available on your system
%
\usepackage{makeidx}         % allows index generation
\usepackage{graphicx}        % standard LaTeX graphics tool
                             % when including figure files
                             
\graphicspath{{./figures/}}
\usepackage{multicol}        % used for the two-column index
\usepackage[bottom]{footmisc}% places footnotes at page bottom

%%% custom commands
\newcommand{\bm}[1]{\boldsymbol{#1}}
\newcommand{\ks}[1]{{(\mathrm{CS})}_{#1}}
\newcommand{\ortvek}[4]{{ }_{(#1)}{\boldsymbol{#2}}^{#3}_{#4} }
\newcommand{\vek}[3]{\boldsymbol{#1}^{#2}_{#3}}
\newcommand{\trmat}[2]{{{ }^{#1}\boldsymbol{T}}_{#2}}
\newcommand{\rotmat}[2]{{{ }^{#1}\boldsymbol{R}}_{#2}}
\newcommand{\rotmato}[2]{{{ }^{#1}\boldsymbol{\overline{R}}}_{#2}}
% Commands for symbol of the residual (full and reduced)
\newcommand{\Res}[0]{\vec{\delta}}
\newcommand{\ResR}[0]{\vec{\psi}}

%%% custom packages
\usepackage[T1]{fontenc}
\usepackage[utf8]{inputenc}
\usepackage{amsmath,amsfonts}
\usepackage{paralist} % for compactitem
\usepackage{siunitx}
\usepackage{url}
\usepackage{newtxtext}       % 
\usepackage[varvw]{newtxmath}       % selects Times Roman as basic font
\makeindex 

\begin{document}

\title*{Inverse Kinematics for Task-Redundancy of Symmetric 3T1R Parallel Manipulators using Tait-Bryan-Angle Kinematic Constraints}
\author{Moritz Schappler}
\institute{%
Leibniz University Hannover, Institute of Mechatronic Systems. \email{moritz.schappler@imes.uni-hannover.de}. Code: \url{github.com/SchapplM/robotics-paper_ark2022_3T1R}.}

%%% Provide shorter versions of title or author list if neceesary:
 \titlerunning{Task-Redundant 3T1R Parallel Manipulators, Tait-Bryan-Angle Kinematics}
 \authorrunning{M. Schappler}

\maketitle
\vspace{-2.5cm} % space above is reserved for author affiliations. Take the space since the one affiliation is on the footer
\abstract{% 10-15 lines long
abstract
} 

%%% Please provide a reasonable number of keywords:
\keywords{Parallel robot, Parallel Manipulator, 3T1R, Task Redundancy, Kinematic constraints, Tait-Bryan angles, Euler angles, Dimensional synthesis.}

\section{Introduction and State of the Art}
\label{sec:introduction}

fundamentals of structural synthesis \cite{Gogu2008,KongGos2007}

symmetric parallel robots \cite{FangTsa2002,HuangLi2003}.

task redundancy mainly used for PM with full mobility in pointing tasks ...

new approach for kinematics model presented by author based on full kinematic constraints and Tait-Bryan angles for rotation beneficial for task redundancy \cite{SchapplerTapOrt2019}

%To encounter the high modeling complexity for 3T2R PMs, the contributions of this paper are
the paper's contributions are
\begin{compactitem}
\item a novel geometric model for the IKP of 3T1R PMs 
\item a gradient-projection scheme for inverse kinematics in task redundancy
\item exemplary application of the model in a combined structural and dimensional synthesis of symmetric 3T1R PM
\end{compactitem}

The remainder of the paper is structured as follows. Sect.~\ref{sec:model} ...
Sect.~\ref{sec:taskred}
The combined structural and dimensional synthesis discussed in Sect.~\ref{sec:synthesis}.
The results of the synthesis are presented in Sect.~\ref{sec:results} and Sect.~\ref{sec:conclusion} concludes the paper.


\section{Kinematic Model for 3T1R Parallel Robots}
\label{sec:model}


\section{Task Redundancy and Inverse Kinematics}
\label{sec:taskred}



\section{Combined Structural and Dimensional Synthesis of 3T1R PMs}
\label{sec:synthesis}


\section{Exemplary Results of the Combined Synthesis for 3T1R PMs}
\label{sec:results}



\section{Conclusion}
\label{sec:conclusion}

conclusion


\begin{acknowledgement}
The author acknowledges the support by the Deutsche Forschungsgemeinschaft (DFG) under grant number 341489206. \textsc{Matlab} code to reproduce the results
is available at GitHub under free license at \url{github.com/SchapplM/robotics-paper_ark2022_3T1R}.
%
\end{acknowledgement}

\bibliographystyle{spmpsci}
\bibliography{references}

\end{document}
